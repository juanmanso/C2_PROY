
	Las mediciones se realizaron con el siguiente instrumental provisto por los laboratorios de la Facultad de Ingeniería de la UBA.

	\graficarPNG{0.8}{instru_M10SP3010E}{Fuente de alimentación \emph{M10SP3010E}.}{fig:M10SP3010E}
	\graficarPNG{0.8}{instru_ADS1102CAL}{Osciloscopio \emph{ATTEN ADS1102CAL}.}{fig:ADS1102CAL}
%	\graficarPNG{0.6}{instru_RIGOL_DS1102E}{Osciloscopio \emph{RIGOL DS1102E}.}{fig:RIGOL_DS1102E}
	\graficarPNG{0.3}{instru_UT30D}{Multímetro \emph{UT30D}.}{fig:UT30D}
	\graficarPNG{0.3}{instru_FG_8002}{Generador de funciones \emph{FG 8002}.}{fig:FG_8002}
	
	% Para medir inductancias usamos PROTOMAX VA511
	\graficarPNG{0.1}{instru_PROTOMAX_VA511}{Medidor de LCR portátil \emph{PROTOMAX VA511}.}{fig:VA511}
