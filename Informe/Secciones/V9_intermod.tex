
Para esta simulación se utilizaron dos señales de entrada superpuestas, una de frecuencia de \SI{100}{\Hz} con amplitud pico \SI{0.4}{\volt} y una de \SI{5}{\kHz} con \SI{0.1}{\volt}, y se mide la distorsión en \SI{5}{\kHz} con los mismos procedimientos de la medición anterior, obteniéndose 
$$ \mathrm{IMD} = 1,4\% $$

%Los resultados obtenidos se muestran en la tabla \ref{tab.intermod}

%\begin{table}[H]
%	\centering
%	\begin{tabular}{ccccc}
%		\toprule
%\multirow{2}{*}{Frecuencia} & \multicolumn{3}{c}{$P_L$} \\ 
%		\cmidrule{2-4}
%			& 4W & 30W & 42W \\
%		\midrule
%		 \SI{11}{\kHz} & \num{2.14} & \num{2.15} & \num{2.15}\\
%		 \bottomrule
%	\end{tabular}
%	\caption{Valores porcentuales de distorsión de intermodulación para distintos valores de frecuencia y potencias sobre la carga.}
%	\label{tab.intermod}
%\end{table}
