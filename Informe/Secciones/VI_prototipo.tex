
	\subsection{Selección de los transistores}
		Durante el proceso de soldado de componentes, se compraron varios transistores iguales con el fin de obtener su $\beta$. Aquellos transistores que tengan $\beta$ similares, se les procedió a ver la relación $V_{BE}$ vs $I_{CQ}$. Para ello, se conectaron los transistores en modo diodo y se les conectó una fuente y una resistencia serie para limitar la corriente. Por los mismos circulará la misma corriente y ahí se mide el $V_{BE}$ de cada uno.

	Fuente 68.

	Cuando hagamos la etapa de salida, hay que estañar las pistas de salida porque quedaron un poco finas y en vez de agrandar de grosor las agrandamos de altura.


	\subsection{Validación del prototipo}

		Mediciones de polarización:

		Vbe del multiplicador de Vbe $= \SI{.55}{\volt}$
		Vce del multiplicador de Vbe $= \SI{2.1}{\volt}$
		110 y 208 millivolt en las re de la carga activa
		Caida en R12 (resistencia del emisor (2.2k) del PD) $= \SI{8}{\volt}$
		
		\Juan{Comprar: cables para los transistores, fichas jack de celular y de guitarra para la placa, disipador, patitas mayores a 2cm de separadores (los plasticos que sostienen a la placa)}
