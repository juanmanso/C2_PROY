
%	Las señales de audio tienen una relación de pico/valor medio elevado, estando la mayor parte del tiempo en valores bajos.

	El trabajo consiste en la confección de un amplificador de audio tanto en su forma teórica como práctica; una vez determinados los valores de los componentes a partir de un análisis teórico y posterior simulación, se procede a armar el mismo de forma discreta y realizar distintas mediciones para compararlas con los datos previamente obtenidos. \\ 

	A continuación se analiza su comportamiento y se detalla cada parte del circuito, explicando el propósito de cada componente y el valor elegido para su óptimo funcionamiento.
