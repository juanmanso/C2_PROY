
	El par diferencial está para compensar el ruido propio de los transistores. La idea es que la naturaleza del ruido es completamente aleatoria y se esparce a lo largo del espectro. Al tener los dos transistores en paralelo, al aparecer un ruido aleatorio en uno de ellos, el otro absorve dicho ruido en contrafase y se cancela. \\

	Suponiendo que los transistores son fuentes de corriente controladas y el ruido se manifiesta como una fuente de corriente, dichos transistores podrán absorver el ruido en la malla que generan entre sí, evitando que se propague a las etapas siguientes.
