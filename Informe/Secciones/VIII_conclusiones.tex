%En el presente informe se logró analizar el circuito \textit{Turner 730}, obteniéndose resultados semejantes por inspección y simulación.

%La respuesta en frecuencia obtenida resultó ser plana dentro del rango de frecuencias audibles e invariante a la potencia disipada. 

%La eficiencia máxima del circuito (71,2\%) resultó cercana a la ideal (78,5\%), aunque nunca llegará a dicho valor ya que el nodo de salida no puede alcanzar los \SI{30}{\V} (por las caídas de tensión de control de los transistores equivalentes).

%La utilización de un par complementario (\textit{Sziklai}) en vez de un \textit{Darlington} reduce notablemente la distorsión armónica. Pero en el cirucito \textit{Turner 730}, uno de los transistores de etapa de salida es \textit{Darlington}. Esta elección de diseño puede deberse a que en los años 70 los transistores \textit{NPN} y \textit{PNP} no presentaban tanta simetría como hoy en día.


	En el presente informe se logró diseñar un amplificador de audio calse G alternativa, obteniéndose resultados por simulación similares a las especificaciones definidas. La comparación se resume en la Tabla \ref{tab.resultados}. La tensión máxima posible sin distorsión logró llegar a \SI{26.1}{\volt} por simulación, esto produce una potencia sobre la carga de \SI{42.25}{\watt}, por lo que se tiene cierta tolerancia con respecto a la potencia nominal especificada.



