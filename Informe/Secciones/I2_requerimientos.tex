El amplificador de audio a diseñar debe cumplir las siguientes especificaciones:

\begin{itemize}
	\item Tensión de alimentación $\pm \SI{30}{\volt}$ con una variación de \SI{5}{\volt}.
	\item Potencia de salida nominal de \SI{39}{\watt} sobre \SI{8}{\ohm} a \SI{1}{\kilo\hertz} con $THD<0,02\%$.
	\item Topología de la etapa de salida Clase G alternativa.
	\item Potencia de salida a \SI{1}{\watt} sobre \SI{8}{\ohm} con $THD<0,01\%$. 
	\item Respuesta en frecuencia \SI{20}{\hertz} a \SI{20}{\kilo\hertz}.
	\item Tensión de \textit{offset} en la salida en continua \SI{5}{\milli\volt}
	\item \textit{Slew Rate} \SI{10}{\volt\per\micro\second}.
	\item IMD 1\%
	\item Impedancia de entrada \SI{100}{\kilo\ohm}
	\item Impedancia de salida del orden de los \SI{}{\milli\ohm}
	\item Factor de amortiguamiento, del orden de 500.
	\item Margen de fase mayor que \SI{45}{\degree}.
\end{itemize}
