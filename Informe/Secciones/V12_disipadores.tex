	A partir de la hoja de datos del \texttt{2SC5198} y de las condiciones del ambiente se tiene:
	\begin{itemize}
		\item  Temperatura de Juntura $T^{\max}_j = \SI{150}{\celsius} $
		\item  Temperatura de Carcasa $T_c = \SI{25}{\celsius} $
		\item  Temperatura de Ambiente $T_a = \SI{40}{\celsius} $
		\item  $\theta_{cs} = \SI{2}{\celsius\per\W}$ 
		\item  Disipación total del \texttt{2SC5198} a $T_{case} = \SI{25}{\celsius}$ se obtiene $\theta_{jc} =  \frac{T_j - T_c}{P_{C_{\max}}} =\SI{1.92}{\celsius\per\W}$ 
	\end{itemize}

		Para no llevar al máximo a la juntura se toma $T_j \leq \SI{120}{\celsius}$, es decir el 70\% del valor máximo. Por otro lado cabe aclarar que la potencia máxima ($P_{C_{\max}}$) que puede soportar el transistor.
		
	Dado estos valores se puede hallar:

	\begin{equation}
		P_{C_{\max}} = \frac{V^2_{CC}}{\pi^2 R_L} = \SI{11.4}{\W}
	\end{equation}

	\begin{equation}
		\theta_{ja} =  \frac{T_j - T_a}{P_{C_{\max}}} =\SI{7.02}{\celsius\per\W}
	\end{equation}
	

	\begin{equation}
		\theta_{ja} = \theta_{jc} + \theta_{cs} + \theta_{sa}
	\end{equation}

	Despejando:

	\begin{equation}
		\theta_{sa} = \theta_{ja} - \theta_{jc} - \theta_{cs} = \boxed{\SI{6.72}{\celsius\per\W}}
	\end{equation}

	Buscando disipadores en \url{www.disipadores.com} se halló que el \texttt{Z36} es el más cercano a las especificaciones requeridas (con $\theta$ alrededor de \SI{8.5}{\celsius\per\W}). Otra alternativa es el disipador con aletas \texttt{Z66} que es más largo y tiene $\theta = \SI{7.5}{\celsius\per\W}$.


