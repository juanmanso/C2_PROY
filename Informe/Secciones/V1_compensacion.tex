	%\Juan{Podríamos poner algo así como que ésto lo hicimos para COMPROBAR que el valor elegido era el correcto.}
Para hallar el valor del capacitor de compensación (presente entre la base y colector del transistor de VAS), se impone una señal cuadrada en la entrada y se observa como resulta la salida. Se realizó una simulación al variar el valor de la capacidad desde \SI{100}{\pico\farad} (curva naranja) hasta \SI{400}{\pico\farad} (curva celeste) de a \SI{100}{\pico\farad}. El resultado se muestra en la Figura \ref{fig:var_cap}.


\HgraficarPNG{0.5}{sim_var_cap_compensacion.png}{Variación de la capacitancia de compensación.}{fig:var_cap}

Por lo tanto para que el circuito esté compensado se eligió un valor comercial de \SI{330}{\pico\farad}.
